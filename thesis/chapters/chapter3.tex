%!TEX root = ../dissertation.tex
%\begin{savequote}[75mm]
%Nulla facilisi. In vel sem. Morbi id urna in diam dignissim feugiat. Proin molestie %tortor eu velit. Aliquam erat volutpat. Nullam ultrices, diam tempus vulputate egestas, %eros pede varius leo.
%\qauthor{Quoteauthor Lastname}
%\end{savequote}

\chapter{Algorithms}

At this point, the dataset is ready for the algorithms development. Recalling that the dataset is composed by comments labeled according to some identified topics, with a three sentiment polarity degrees, plus an additional relevancy flag, that makes the final problem a four-labels classification task. This problem has been faced in different ways, that will be explained in the Chapter 5. In this Chapter, will be argued the algorithms that will be involved in the classifications and the choices made for the development. The main pipeline remain the same of state of the art text classification problems, that has been presented in Chapter 2, but some implementation choices are needed in order to make the algorithms specific for the automotive domain. Since (almost) same algorithms can be run on different datasets, it has been made a comparison using the Twitter's Airline dataset, and the differences on implementation have been explained later on.\\
All code has been developed in Python 3.7 (\url{https://www.python.org/downloads/release/python-370/}) using the Jupyter Notebook (\url{https://jupyter.org/}), that is a web application that allows to create live Python code. 
The code is available on the repository \url{https://github.com/elRava/master_thesis_sentiment_analysis/}.
% TODO dare occhiata se tenere link repo


\section{SVM and Logistic Regression Classification}

The simpler algorithms implemented in this work, follow the state of the art pipeline for text classification. Since the algorithms are supposed to work on very different environments, they present some differences, especially on preprocessing stage. The same algorithm is used for both binary classification for relevance test, and multilabel sentiment classification.\\


\subsection{Twitter Preprocessing}

Preprocessing for Twitter's dataset handles Twitter's specific tokens and normalize them in order to treat them in a standard way. Some functions are optional in the sense that the preprocessing has been made parametric and it is possible to choose if do them or not. The steps are the followings:

\begin{itemize}
	\item Transformation text to lowercase;
	\item Replace two or more dots with space, in order to remove useless punctuation;
	\item For every token, remove spaces, " and ';
	\item Optional substitution of the URLs with "URL";
	\item Optional replacing user mentions (@user) with "USER\_MENTION";
	\item Optional removing "\#" from hashtags (\#hashtag);
	\item Optional removing retweet character ("RT");
	\item Optional grouping emoticons into positive and negative ones and replacing them with "EMO\_POS" and "EMO\_NEG". Positive emoticons are: :), : ), :-), (:, ( :, (-:, :'), :D, : D, :-D, xD, x-D, XD, X-D, <3, :*, ;-), ;), ;-D, ;D, (;,  (-;, while Negative ones are: :-(, : (, :(, ):, )-:, :,(, :'(, :"( ;
	\item Removing residual multiple spaces.
\end{itemize}



\subsection{Italian Automotive Dataset Preprocessing}

As in Twitter's preprocessing, the goal is to reduce noise by handling domain specific terms. Also in this case the preprocessing has been made parametric. The steps are the followings:

\begin{itemize}
	\item Encoding correction;
	\item Punctuation removal;
	\item Transformation text to lowercase;
	\item Removal of character repetitions: keep at maximum three consecutive character repeated;
	\item Replaced question marks and consecutive question marks with respectively "QMARK" and "MULTI\_QMARK";
	\item Replaced exclamation marks and consecutive question marks with respectively "EMARK" and "MULTI\_EMARK";
	\item Replacing URLs with "URL";
	\item Replacing HTML picture tags with "IMG";
	\item Stopwords removal;
	\item Optional replacing brands with "BRAND" and car models with "MODEL": since the classifications are considered brand independent, it comes the idea of replacing all brand names with a common token. The dictionary of all manufacturers and models has been scraped from \url{https://www.auto-data.net}, since a complete one has not be found;
	\item Replacing speed metrics with "SPEED": since the car independent intentions, it does not matter the actual speed values, for instance 150 km/h may be good for a city car, but not so good for a sport car. Moreover, it depends on the context, so the decision of ignore the numerical value. For the same reason also consumption metrics have been replaced with "CONSUMPTION", weights metrics with "WEIGHT" and power metrics with "POWER";
	\item Replaced distances with "DISTANCE";
	\item Replaced numbers with three or more digits with "NUMBER".
\end{itemize}

All operation have been implemented exploiting regular expressions from the re library (\url{https://github.com/python/cpython/blob/3.7/Lib/re.py}) and common natural language processing operation from nltk library (\url{https://www.nltk.org/}).


\subsection{Classification}

Following the preprocessing, it is applied the stemmer. The result of the whole preprocessing phase on a tweet is:

\begin{description}
	\item \textit{@united I do not see where it talks about military baggage fees. Can you please guide me. Thanks}
	\item \textit{USER\_MENTION see talk militari baggag fee  pleas guid  thank
	}
\end{description}

and on the italian automotive dataset is:

\begin{description}
	\item \textit{Sono reali calcolati nel arco del tutto anno nel estate qualcosa in più causa gomme di 17" e climatizzatore nel inverno un po di meno. Per quanto riguarda le autostrade quelle che percorro io principalmente la A4 e molto congestionata cosi spesso la media e 110-115 km/h che ovviamente influisce positivamente a i consumi. Ma quello che mi piace di più è assenza dei guasti. Sulla vecchia Accord il primo guasto lo ho avuto a 200000 km si è rotto il termostato della clima. Ogni tanto faccio giro di altri forum e leggo delle turbine rotte catene di distribuzione progettate male iniettori fatti male mah nel 2015 per me sono le cose incomprensibili . Con tutti gli difetti che può avere preferisco la Honda. }
	
	\item \textit{real calcol arco anno MODEL qualcos caus gomm MODEL climatizz invern po men riguard autostrad percorr principal MODEL molt congestion cos spess med MODEL SPEED ovvi influ posit consum piac assenz guast vecc MODEL prim guast DISTANCE rott termost clim ogni MODEL gir altri forum legg turbin rott caten distribu progett mal iniettor fatt mal mah NUMBER me cos incomprens difett può aver prefer BRAND}
\end{description}

For all experiments on the italian automotive dataset, I firstly focused on just one topic, then the same procedure has been extended to all others. Since this dataset has not just one text, but it contain also the quote, the classification should involve both texts. It has been implemented joining the texts and, in a parametric way, adding a suffix to each word like "\_TEXT" or "\_QUOTE" weather a word is belonging to the text or to the quote.

For vectorization it has been chosen the TF-IDF method. Classification involved either Support Vector Machine classificator and Logistic Regression classificator for different experiments, both from scikit-learn library (\url{https://scikit-learn.org/}).


An important task for model selection consists on hyperparameters optimization. Hyperparameters are model parameters that must be chosen before the learning process begins. In general, different algorithms require different hyperparameters, and some others also none. The search of the optimum parameters has been implemented using a grid search, which requires a proper list of parameters and a list of testing values. It searches among all, the parameters the ones that, when the model is trained with those, reaches the best score. The reference score, of course, must be defined previously and depends on the classification problem.
Feature selection has been performed in the same way for both SVM and Logistic Regression. After a first optimization involving all features, found weights of the classifier are ranked with respect to the absolute value. Then, after some tries it is defined an adequate cutoff value which selects the features to keep. The final model is obtained by train again the algorithm keeping only the selected features.


\section{BPEF Revisitation}

The second algorithm employed for text classification is a revisitation of the BPEF algorithm described in Chapter 2. It has been chosen this algorithm because of its proved good performance and stability in Twitter sentiment analysis, and above all, the capability to adaptation to slightly different problems. The adaptation procedure will be described in the followings paragraphs.

\subsection{Parametric Model}

As shown in Section 2.4.1, Bootstrap Ensemble Framework is based upon a parametric model that involves dataset, feature and classifier parameters. Every type of parameters had been modified from the original paper and it has been adapted for the problem of this work.

\subsubsection{Dataset Parameters}

In the original BPEF algorithm, the target dataset is integrate with other dataset of similar domain. The purpose is to enrich the dictionary of common saying and common patterns that are used for sentiment classification in order to make the classifier more stable. Actually, the italian automotive dataset already contains a comments with a large variety of expressions, and most important, it can be considered also an union of different datasets, in fact it contain comments from very different car brands, that cause variety on subjects and patterns. For instance, suppose that the target classification is a the set of comments of sports cars, but since the dataset contain comments of every type of automotive vehicles, then off-topic ones can be considered as an integration, that in BPEF algorithm consisted in the aggregation of multiple datasets. In conclusion, because of these reasons, dataset parameters have just been removed.


\subsubsection{Feature Parameters}

\subsubsection{Classifier Parameters}






\section{Cascade Classification}
















tsa svm

tsa bpef

my svm rel

my logreg rel

my svm snt

my svm 4lab

my bpef snt

my cascade logreg svm

my cascade logreg bpef



