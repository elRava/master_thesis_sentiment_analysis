%!TEX root = ../dissertation.tex
%\begin{savequote}[75mm]
%This is some random quote to start off the chapter.
%\qauthor{Firstname lastname}
%\end{savequote}

\chapter{Dataset}

In this chapter it will be presented the dataset that has been utilized for this work. As previously mentioned, the main focus is asked to be italian automotive forums, so the first phase consisted in the search of reliable resources from which retrieve the needed information. Information retrieval has been made by crawling the founded web forums. After getting all the information, the final phase for what concerns the dataset consists on annotate all the forums' comments on the base of diverse topics. \\
In the following sections the whole procedure has been presented in detail.

\section{Dataset Retrieval}

Sentiment analysis is a practical natural language processing problem, most commonly faced involving machine learning techniques. As discussed in detail in Chapter 2, machine learning algorithms need a set of labeled data that represent the environment where the algorithm is supposed to work on. As good sampled data represent the true data distribution, as good the model should work on the real environment, so the dataset retrieval is fundamental for the final outcomes.\\
This work is focused on italian automotive forums, so the first important phase is to identify some reliable resources where to extrapolate the needed information. 


\subsection{Forums' List}

Web forums are websites (or sections of websites) that allow visitors to communicate with each other by posting messages. Most forums allow anonymously visitors to view forum contents, except for registration-required ones. Forums are available for all kinds of topics, from software to health, and like in this case, automotive. Forums are composed by user-generated content, so they are full of users' opinions then, as just mentioned, they are good environments to acquire datasets for sentiment analysis. Forums are commonly subdivided in "areas" or "rooms" which are thematic macro partitions. Rooms are again subdivided into "sections", which are discussions' categories belonging to the same thematic area. Finally, sections contain numerous "threads", which are the actual discussions.\\



For this work, target resources are identified in italian automotive web forums or blogs with possibility of discussion, with an active community. The identified resources are: Quattroruote(\url{https://forum.quattroruote.it/}), Autopareri (\url{https://www.autopareri.com/forum}), Bmwpassion (\url{https://www.bmwpassion.com/forum/}), Porschemania (\url{http://www.porschemania.it/discus/}) and HDmotori (\url{http://www.hdmotori.it/}). 
Some statistics are summarized in Table \ref{table:forum-list}.

\begin{table}[ht]
	\renewcommand{\arraystretch}{2.5}
	\centering
	\begin{tabular}{| >{\centering\bfseries}m{3cm} | >{\centering}m{1in} | >{\centering}m{1in} | >{\centering}m{1in} | >{\centering\arraybackslash}m{1in} | } 
		\hline
		\textbf{Resource} & \textbf{Type} & \textbf{\# Threads} & \textbf{\# Messages} & \textbf{\# Users} \\ [.2cm]
		\midrule
		Quattroruote & Forum & 121.366 & 2.413.452 & 70.707 \\ [.2cm]
		\hline
		Autopareri & Forum & n.a & 2.146.532 & 34.963 \\ [.2cm]
		\hline
		Bmwpassion & Forum & 349.259 & 7.909.297 & 78.608
		 \\ [.2cm]
		\hline
		Porschemania & Forum & n.a & n.a & n.a \\ [.2cm]
		\hline
		HDmotori & Blog & n.a. & n.a. & n.a. \\ [.2cm]
		\hline
	\end{tabular}
	\caption{Forums' statistics updated in date 15/08/2019.}
	\label{table:forum-list}
\end{table}

% TODO eventualmente spiegare qualcosa di tutti i forum

All resources except for Porschemania are brand independent, which means that are treated discussions about every automotive brands independently, while the last one is focused on Porsche \footnote{Porsche is a German sports car manufacturer located in Zuffenhausen in Stuttgart, founded in 1931 by Ferdinand Porsche.} discussions.\\
The dataset will be composed as a list of comments, so the idea is to download as mush information as possible, that is common to each resource, in order to represent precisely the environment in a structured way like a Comma Separated Values (CSV) file.

\subsection{Comments' Information}

When decided the resources' list, it is a good practice to identify the common information contained in forums' comments that are going to be downloaded.


\subsection{Crawl}




\section{Dataset Annotation}

\section{Statistics}

