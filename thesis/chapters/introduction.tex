%!TEX root = ../dissertation.tex
\chapter{Introduction}
\label{introduction}

Opinions are thoughts that all humans do, that also influence everyone's behaviors. Our beliefs and choices are conditioned upon other people see the world, for this reason, when we need to make a decision, we tend to see the opinion of others. Opinions and emotions are the subject of study of the \textit{sentiment analysis}, also called \textit{opinion mining}.\\
With the rapid growth of the social networks and online forums, people use to talk about every kind of subject, depending obviously on the website's domain. Nowadays, for instance, if one wants to buy a product, is no longer limited to ask opinions only to someone close to him, but there are lot of customers' discussions on public available online resources. Similarly, for an organization it is no more mandatory to conduct surveys and polls in order to gather customers' opinions, because of the plenty of information gathered on social media. For companies, obtaining information about people's opinion, may take advantages against competitors, directing the company's choices to the customers' will. Before automatic sentiment analysis tools, marketing employees needed to manually scraping the web pages searching for opinions related to their brands, or conducting surveys and opinion polls. This solution is very time consuming, and need a lot effort, so it is not simple nor efficient. Designing an automated service that searches people's opinions on some given resources makes all the job at marketing side, simpler and more focused on decision making rather than opinions' gathering. As long as sentiment analysis have been studied, it spread to almost every possible domain, from product reviews, services, healthcare, financial services, or political elections.\\
Technically speaking, sentiment analysis is a branch of the \ac{NLP} that started to be studied intensively after the year 2000, and the main reason is that online resources were not already massively used before these years \cite{Liu:2012:SAO:3019323}. In general, sentiment analysis had been investigated mainly at three levels: document, sentence and entity levels. Document level means that the task consists on classify whether a whole opinion document express a positive or negative sentiment about the product. This level of analysis assumes that each sentence express an opinion about a product.  The task at sentence level is used for determining whether each sentence express a positive or negative opinion. Finally, entity level is used to discover what people like or dislike. It consists on the fact that users' opinions contain information about the sentiment and the target (an opinion without a target may no be useful).\\
Most existing techniques for sentiment classification involve supervised learning, where the problem is seen as a three (or eventually more) labels classification, which are the sentiment polarities "positive", "neutral" and "negative". Supervised learning is a subtask of machine learning that needs a dataset of already classified data, in order to derive a generalization between input (in this case the text) and the output (in this case the sentiment polarity). The issue of this methodology is that it requires the initial data, and those must be specific for the task where the classifier is supposed to work. There are numerous dataset specific for sentiment analysis, most of them gathered from social media, but sometimes data there are not available, so it is mandatory to design a way to build a proper one.\\
This work is focused on designing a sentiment analysis tool for evaluate users' opinion on online Italian automotive forums. There exist, actually, a market on sentiment analysis related to the automotive area, where it is studied the sentiment and the perception of the people about certain arguments on that field. These years, for instance, are the ones where the electrical mobility is starting to become popular, so car manufacturers want to know users' perception in order to make choices for having an advantage on the market against competitors. Another example may be the customers' satisfaction for a luxury car manufacturer, which obviously has the perceived quality as one of the main steady points for the brand. For gathering users' opinions there exists numerous web forums, where are treated every kind of discussions about literally every kind of topic in the automotive world. However, as said, manual scraping is not sustainable for the quantity of the data.\\
In the automotive field there exist many thematic areas where to search for people's opinion, for instance the engine, the cars' exteriors, the maintenance, etc. The focus actually depends mostly on the customer's request, which can be interested in some particular aspects, as previously mentioned. For this reason, it is important to work with some experts in the car brand customer, in order to understand their needs and their particular aspects that they are going to analyze.\\
This work has been made in collaboration with Reply Technology (\url{https://www.reply.com/technology-reply/}), a company that takes part on the Reply society. Reply, founded in 1996 in Turin by Mario Rizzante, is a consulting company that designs ans implements solutions based on innovative technologies with the goal of help customers to enhancing their values. Activities in Reply may be classified with respect to the target market: telco and media, banks and assurances, industries and services, energy and public administration. Technology Reply is one company of the group, specialized in data management with Oracle technology. This thesis came from the request from the car manufacturer Porsche, to conduct an analysis on people's opinion about its brand. All the presented work is the design a part of a proposal of solution for intercepting the request of the client.\\
Starting from data gathering, it has been developed a strategy for creating a suitable dataset, gathering comments on any type from most popular Italian forums specialized on cars and cars' manufacturers discussions. In a second phase, the gathered data must be manually labeled involving a sort of crowdsourcing, in order to obtain a dataset that can be used for designing supervised leaning models.\\
Naturally, before starting with the actual work, it has been studied the related state of the art, starting from the basics of machine learning, more precisely the supervised learning subfield, with all the theoretical information needed to understand and use consciously the implementation of the algorithms. From state of the art, have been studied some of the most powerful sentiment analysis models, for which have been highlighted pros and cons, and finally starting from a comparison, has been chosen a model that is supposed to work well for the problem of this work. The chosen model will be the so called \ac{BPEF}, for which it will be presented an implementation with all the steps needed for adapting it from the original paper's task to the one for this work.\\
Successively, it will be studied the performance of the algorithm, compared to a baseline approach, which is an implementation that involves \acl{SVM} classifier, with a custom preprocessing for the domain specific classification. In this way, it will be shown the better performance of the \ac{BPEF} model, that always overcomes the baseline approach. The sentiment classifiers models will be then included into a cascade classifier for a four-label classification that involves also the relevance property of the text with respect to a particular topic, making so an aspect-based sentiment classification.\\
Finally, after the design of the final model, it will be presented a use case of the classifier, involving a business intelligence software for data visualization, where can be conducted some analysis on the data, and verify the reliability of the solution.\\
The thesis follows this organization:\\
In the \hyperref[state-of-the-art]{\textcolor[RGB]{35,103,148}{SECOND CHAPTER}} it will be presented the state of the art for what concerns the sentiment analysis, starting from the basics of machine learning, to the actual sentiment analysis tools.\\
In the \hyperref[dataset]{\textcolor[RGB]{35,103,148}{THIRD CHAPTER}} it will be presented the followed strategy for collecting the sentiment analysis dataset for this work, presenting also some preliminary statistics, making also comparisons with other existing datasets .\\
In the \hyperref[algorithms]{\textcolor[RGB]{35,103,148}{FOURTH CHAPTER}} it will be presented the implementation of the algorithms that have been designed for this work, both baseline and \ac{BPEF}, and all the strategies for introducing the aspect based sentiment analysis.\\
In the \hyperref[experiments]{\textcolor[RGB]{35,103,148}{FIFTH CHAPTER}} it will be discussed all the experiments made with the designed models, discussing also the choices made for building the cascade classifier.\\
In the \hyperref[industrial-use-case]{\textcolor[RGB]{35,103,148}{SIXTH CHAPTER}} it will be presented a use case of the designed sentiment analysis model, involving the use of a business intelligence tool.\\
In the \hyperref[conclusion]{\textcolor[RGB]{35,103,148}{SEVENTH CHAPTER}} it will be expressed the conclusions about the work done, making some considerations about the obtained results and discussing some future works related to this work.\\

