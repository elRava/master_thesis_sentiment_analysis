%!TEX root = ../dissertation.tex
\chapter{Conclusions and Future Works}
\label{conclusion}

We conclude this work making some consideration on the obtained results with the presented algorithms for sentiment analysis on Italian automotive forums. Along with these results, we propose some improvements to enhance the performance of the models in order to make the future results more reliable. Finally, will be proposed a solution in order to integrate this sentiment analysis tool in a real world case.

\section{Results of the Sentiment Analysis Models}

The goal of this work was to apply sentiment analysis to the automotive dataset, on online Italian forums comments. The chosen strategies involved machine learning approaches, and since it does not already exist suitable datasets, the first approach was to collect one. With some manually designed scripts, have been crawled some popular forums, obtaining almost 1,200,000 comments. Even if the subject of the sentiment was a particular manufacturer, comments have been taken from every discussion of any manufacturer, in order to inject some noise, which is useful to avoid biases. From a preliminary look at the topic most treated into the forums, has been selected a list of arguments, for which every comment must be labeled, selecting a sentiment polarity between "very positive", "positive", "neutral", "negative" and "very negative", plus an additional "irrelevant" for saying that in the comment the argument is not treated. In a second moment, the polarities were grouped into a three-degrees sentiment scale. All tests were made for the class "engine", since is is one of the most populated (even if not plenty of data at all), and with explicit references to the engine, that are supposed to make the classification easier.\\
In this work it has been designed a two-stages cascade classifier, in all its parts, and it was compared with a one-step classifier, for detecting the pertinence about some topics and eventually the sentiment of automotive comments.\\
Before starting with the crawled dataset, have been made some tests on a sentiment analysis Twitter's dataset, that actually show great performance compared to the state of the art. A baseline approach involving \aclp{SVM} classifier reached around 70\% accuracy, which is a remarkable result, but improved with the designed revised \aclp{BPEF} model, that reached around 75\% accuracy, which is close to state of the art results. In all cases, features selection improved the overall performance, which means that the used techniques were able to select most important components that carried the sentiment information. These experiments served as benchmark to test the goodness of the implementation of the model, before to design the main model for the Italian dataset.\\
When it was shown that implemented models worked, it was designed a baseline approach for classification of the "engine" class, and then an improved cascade classifier. The first reached poor results, due to the lack of comments, but mostly because of the high imbalances to the "imbalance" class that inducted a high bias, that made the "negative" polarity actually ignored. For this reason, the idea to split the overall classification into two stages: a first that detects if an argument is treated, and the second that classifies the sentiment. Splitting the problems into two phases, the two classifiers can be trained with less imbalance classes, and the performance showed a big improvement: the scores present a huge improvement with respect to the baseline approach, but most importantly the reliability of the solution. With these facts, the choice of splitting the problem into two smaller ones was the right one for this problem. % todo classificazione ocn le altre classi



\section{Enhancements}


\section{Real World System}

